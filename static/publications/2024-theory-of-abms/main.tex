\documentclass{article}



\usepackage{arxiv}

\usepackage[utf8]{inputenc} % allow utf-8 input
\usepackage[T1]{fontenc}    % use 8-bit T1 fonts
\usepackage{hyperref}       % hyperlinks
\usepackage{url}            % simple URL typesetting
\usepackage{booktabs}       % professional-quality tables
\usepackage{amsfonts}       % blackboard math symbols
\usepackage{nicefrac}       % compact symbols for 1/2, etc.
\usepackage{microtype}      % microtypography
\usepackage{graphicx}
\usepackage{natbib}
\usepackage{doi}



\title{
    A Mathematical Language of\\
    Cellular Agent-Based Models
}

%\date{September 9, 1985}	% Here you can change the date presented in the paper title
%\date{} 					% Or removing it

\author{
    \href{https://orcid.org/0009-0001-0613-7978}{\includegraphics[scale=0.06]{orcid.pdf}\hspace{1mm}Jonas Pleyer}
    \thanks{\href{https://jonas.pleyer.org}{jonas.pleyer.org}} \\
	    Freiburg Center for Data-Analysis and Modeling\\
	    University of Freiburg\\
	    \texttt{jonas.pleyer@fdm.uni-freiburg.de} \\
	%% examples of more authors
	\And
	\href{https://orcid.org/0000-0000-0000-0000}{\includegraphics[scale=0.06]{orcid.pdf}\hspace{1mm}Christian Fleck} \\
	    Freiburg Center for Data-Analysis and Modeling\\
	    University of Freiburg
	    % Santa Narimana, Levand \\
	    % \texttt{christian.fleck@fdm.uni-freiburgd.e} \\
}

% Uncomment to remove the date
%\date{}

% Uncomment to override  the `A preprint' in the header
%\renewcommand{\headeright}{Technical Report}
%\renewcommand{\undertitle}{Technical Report}
\renewcommand{\shorttitle}{\textit{arXiv} Template}

%%% Add PDF metadata to help others organize their library
%%% Once the PDF is generated, you can check the metadata with
%%% $ pdfinfo template.pdf
\hypersetup{
pdftitle={A Mathematical Language of Cellular Agent-Based Models},
pdfsubject={q-bio.NC, q-bio.QM},
pdfauthor={Jonas Pleyer, Christian Fleck},
pdfkeywords={},
}

\begin{document}
\maketitle

\begin{abstract}
\end{abstract}


% keywords can be removed
\keywords{Agent-Based \and Cell}


\section{Introduction}

\section{Properties of Individual-Based Descriptions in Multicellular Systems}
\label{sec:properties}

\begin{table}
	\caption{Aspects of cellular systems}
	\centering
	\begin{tabular}{lcl}
		\toprule
		% \multicolumn{2}{c}{Part}                   \\
		% \cmidrule(r){1-2}
        Aspect                  &Type   &Examples\\
        \midrule
        Spatial Representation  &C      &elongated rods, soft spheres, vertex-like\\
        Intracellular Reactions &C      &signalling pathways,stress response\\
        Movement                &C      &neighbour exchange, motility, chemotaxis, directed random walk\\
        Cycle                   &C      &differentiation, division, (phased) death, mesenchymal-epithelial transition\\
		\midrule
        Forces                  &CC     &adherent forces, friction, de-adhesion\\
        Reactions via Contact   &CC     &communication between plant-cells via plasmodesmata, gap-junctions\\
        Neighbour sensing       &CC     &effective abstraction\\
        Fusion                  &CC     &homotypic, heterotypic\\
        \midrule
        External Forces         &DC     &blood flow near heart-valves, microfluidic devices\\
        Extracellular Processes &D, DC  &diffusion of morphogens, transport of signalling molecules\\
		\bottomrule
	\end{tabular}
	\label{tab:table}
\end{table}

\subsection{Intracellular Processes}
\label{subsec:properties-cell}

\subsection{Cell-Cell Interactions}
\label{subsec:properties-cell-cell}
\subsubsection{Forces}
\label{subsec:properties-cell-cell-forces}

\subsection{Domain-Cell Interactions}
\label{subsec:properties-domain-cell}

% \subsection{Citations}
% Citations use \verb+natbib+. The documentation may be found at
% \begin{center}
% 	\url{http://mirrors.ctan.org/macros/latex/contrib/natbib/natnotes.pdf}
% \end{center}
% 
% Here is an example usage of the two main commands (\verb+citet+ and \verb+citep+): Some people thought a thing \citep{kour2014real, hadash2018estimate} but other people thought something else \citep{kour2014fast}. Many people have speculated that if we knew exactly why \citet{kour2014fast} thought this\dots
% 
% \subsection{Figures}
% See Figure \ref{fig:fig1}. Here is how you add footnotes. \footnote{Sample of the first footnote.}
% 
% \begin{figure}
% 	\centering
% 	\fbox{\rule[-.5cm]{4cm}{4cm} \rule[-.5cm]{4cm}{0cm}}
% 	\caption{Sample figure caption.}
% 	\label{fig:fig1}
% \end{figure}
% 
% \subsection{Tables}
% See awesome Table~\ref{tab:table}.
% 
% The documentation for \verb+booktabs+ (`Publication quality tables in LaTeX') is available from:
% \begin{center}
% 	\url{https://www.ctan.org/pkg/booktabs}
% \end{center}
% 
% 
% \begin{table}
% 	\caption{Sample table title}
% 	\centering
% 	\begin{tabular}{lll}
% 		\toprule
% 		\multicolumn{2}{c}{Part}                   \\
% 		\cmidrule(r){1-2}
% 		Name     & Description     & Size ($\mu$m) \\
% 		\midrule
% 		Dendrite & Input terminal  & $\sim$100     \\
% 		Axon     & Output terminal & $\sim$10      \\
% 		Soma     & Cell body       & up to $10^6$  \\
% 		\bottomrule
% 	\end{tabular}
% 	\label{tab:table}
% \end{table}
% 
% \subsection{Lists}
% \begin{itemize}
% 	\item Lorem ipsum dolor sit amet
% 	\item consectetur adipiscing elit.
% 	\item Aliquam dignissim blandit est, in dictum tortor gravida eget. In ac rutrum magna.
% \end{itemize}


\bibliographystyle{unsrtnat}
\bibliography{references}  %%% Uncomment this line and comment out the ``thebibliography'' section below to use the external .bib file (using bibtex) .


%%% Uncomment this section and comment out the \bibliography{references} line above to use inline references.
% \begin{thebibliography}{1}

% 	\bibitem{kour2014real}
% 	George Kour and Raid Saabne.
% 	\newblock Real-time segmentation of on-line handwritten arabic script.
% 	\newblock In {\em Frontiers in Handwriting Recognition (ICFHR), 2014 14th
% 			International Conference on}, pages 417--422. IEEE, 2014.

% 	\bibitem{kour2014fast}
% 	George Kour and Raid Saabne.
% 	\newblock Fast classification of handwritten on-line arabic characters.
% 	\newblock In {\em Soft Computing and Pattern Recognition (SoCPaR), 2014 6th
% 			International Conference of}, pages 312--318. IEEE, 2014.

% 	\bibitem{hadash2018estimate}
% 	Guy Hadash, Einat Kermany, Boaz Carmeli, Ofer Lavi, George Kour, and Alon
% 	Jacovi.
% 	\newblock Estimate and replace: A novel approach to integrating deep neural
% 	networks with existing applications.
% 	\newblock {\em arXiv preprint arXiv:1804.09028}, 2018.

% \end{thebibliography}


\end{document}

