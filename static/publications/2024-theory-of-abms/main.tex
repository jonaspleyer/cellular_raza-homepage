\documentclass{article}



\usepackage{arxiv}

\usepackage[utf8]{inputenc} % allow utf-8 input
\usepackage[T1]{fontenc}    % use 8-bit T1 fonts
\usepackage{hyperref}       % hyperlinks
\usepackage{url}            % simple URL typesetting
\usepackage{booktabs}       % professional-quality tables
\usepackage{amsmath}
\usepackage{amsfonts}       % blackboard math symbols
\usepackage{nicefrac}       % compact symbols for 1/2, etc.
\usepackage{microtype}      % microtypography
\usepackage{graphicx}
\usepackage{natbib}
\usepackage{doi}



\title{
    A Mathematical Language of\\
    Cellular Agent-Based Models
}

%\date{September 9, 1985}	% Here you can change the date presented in the paper title
%\date{} 					% Or removing it

\author{
    \href{https://orcid.org/0009-0001-0613-7978}{\includegraphics[scale=0.06]{orcid.pdf}\hspace{1mm}Jonas Pleyer}
    \thanks{\href{https://jonas.pleyer.org}{jonas.pleyer.org}} \\
	    Freiburg Center for Data-Analysis and Modeling\\
	    University of Freiburg\\
	    \texttt{jonas.pleyer@fdm.uni-freiburg.de} \\
	%% examples of more authors
	\And
	\href{https://orcid.org/0000-0000-0000-0000}{\includegraphics[scale=0.06]{orcid.pdf}\hspace{1mm}Christian Fleck} \\
	    Freiburg Center for Data-Analysis and Modeling\\
	    University of Freiburg
	    % Santa Narimana, Levand \\
	    % \texttt{christian.fleck@fdm.uni-freiburgd.e} \\
}

% Uncomment to remove the date
%\date{}

% Uncomment to override  the `A preprint' in the header
\renewcommand{\headeright}{Preprint}
%\renewcommand{\undertitle}{Technical Report}
\renewcommand{\shorttitle}{A Mathematical Language of Cellular Agent-Based Models}

%%% Add PDF metadata to help others organize their library
%%% Once the PDF is generated, you can check the metadata with
%%% $ pdfinfo template.pdf
\hypersetup{
pdftitle={A Mathematical Language of Cellular Agent-Based Models},
pdfsubject={q-bio.NC, q-bio.QM},
pdfauthor={Jonas Pleyer, Christian Fleck},
pdfkeywords={},
}

\begin{document}
\maketitle

\begin{abstract}
\end{abstract}


% keywords can be removed
\keywords{Agent-Based \and Cell}


\section{Introduction}
\label{sec:introduction}

\subsection{Properties of Multicellular Systems}
\label{subsec:introduction-properties}

\begin{enumerate}
    \item Stochastic -> Irreversible
    \item Dissipative (take up energy, non-equilibrium, decrease entropy)
\end{enumerate}


\subsection{Individual-Based Descriptions of Cellular Agent}
\label{subsec:introduction-individual-descriptions}

\begin{table}
	\caption{Aspects of cellular systems}
	\centering
	\begin{tabular}{ll}
		% \multicolumn{2}{c}{Part}                   \\
		% \cmidrule(r){1-2}
		\toprule
        Aspect                  &Examples\vspace{0.5em}\\
        \it{(C) Cellular}\\
        \midrule
        Spatial Representation  &elongated rods, soft spheres, vertex-like\\
        Intracellular Forces    &hexagonal shape\\
        Intracellular Reactions &signalling pathways,stress response\\
        Movement                &neighbour exchange, motility, chemotaxis, directed random walk\\
        Cycle                   &differentiation, division, (phased) death, mesenchymal-epithelial transition\vspace{0.5em}\\
        \it{(CC) Cell-Cell Interactions}\\
		\midrule
        Interaction Forces      &adherent forces, friction, de-adhesion\\
        Reactions via Contact   &communication between plant-cells via plasmodesmata, gap-junctions\\
        Neighbour sensing       &phenomenological abstraction\\
        Fusion                  &homotypic, heterotypic\vspace{0.5em}\\
        \it{(DC) Domain-Cell Interactions}\\
        \midrule
        External Forces         &blood flow, microfluidic devices\\
        Boundary Effects        &reflection at petri-dish surface, containment of ligands, adhesion\\
        Extracellular Processes &diffusion of morphogens, transport of signalling molecules\vspace{0.5em}\\
        \it{(O) Other}\\
        \midrule
        External Controller     &introduce drug, remove cells, refresh nutrients\\
		\bottomrule
	\end{tabular}
	\label{tab:table}
\end{table}

\section{Motivation}
\label{subsec:motivation}
In the previous section, we have determined aspects of cellular systems which we aim to describe
with suitable abstractions.
As we have already shown, these aspects can be grouped into 3 categories.
The first category only considers a single cell by itself while the second and third consider
interactions with other cells and the external environment respectively.
Lastly, we allow an external controller to make changes to the experimental setup in accordance with
the experimental reality.

\subsection{Intracellular Representations and Processes}
\label{subsec:abstractions-cell}
\subsubsection{Mechanics - Spatial Representation, Intracellular Forces \& Movement}
\label{subsubsec:abstractions-cell-mechanics}
\subsubsection{Intracellular Reactions}
\label{subsubsec:abstractions-cell-reactions}
\subsubsection{Cycle}
\label{subsubsec:abstractions-cell-cycle}

\subsection{Cell-Cell Interactions}
\label{subsec:abstractions-cell-cell}
\subsubsection{Forces}
\label{subsec:abstractions-cell-cell-forces}
Forces acting between cells are often described in terms of interaction potentials $V(x_i, x_j)$
where $x_i,x_j$ are the position of two cells.
\begin{equation}
    F_i = \nabla_{x_i}V(x_i, x_j)
\end{equation}
If $V$ only depends on the difference between the two positions $x_i-x_j$, we obtain an
antisymmetric expression in $i,j$.
\begin{align}
    F_i &= \nabla_{x_i}V(x_i-x_j)\\
    &= -\nabla_{x_j}V(x_i-x_j)
\end{align}
However, this assumption is only valid for point-like objects without any additional structure.
Consider the example where a cell is represented by two points $(p_1, p_2)$ which are connected with
a spring of length $l$ and strength $D$.
This may be a simplified model for an elongated bacteria.
We can write down the force contribution from intracellular forces
\begin{align}
    F_{\text{intra},i} = \begin{bmatrix}
        - D (|p_1-p_2|-l)\\
        D (|p_1-p_2|-l)
    \end{bmatrix}
\end{align}
We can assume that the two vertices of one cell $p_{1,i},p_{2,j}$ are interacting with both vertices
of the other.
This reslts in a force term of the form
\begin{align}
    F_{\text{extra},i} &= \begin{bmatrix}
        F(p_{1,i}, p_{1,j}) + F(p_{1,i}, p_{2,j})\\
        F(p_{2,i}, p_{1,j}) + F(p_{2,i}, p_{2,j})
    \end{bmatrix}\\
    F_{\text{extra},j} &= \begin{bmatrix}
        F(p_{1,j}, p_{1,i}) + F(p_{1,j}, p_{2,i})\\
        F(p_{2,j}, p_{1,i}) + F(p_{2,j}, p_{2,i})
    \end{bmatrix}
\end{align}
and thus we can clearly see that $F_{\text{extra},i}\neq - F_{\text{extra},j}$ even if
$F(a,b)=-F(b,a)$.
This means that we have to rely on a method which calculates both the forces acting on cell $i$ and
$j$ simultanously.
To be able to model friction forces (although rare) we additionally take the velocity $v_i$ into
account.
Further, we allow additional information $\sigma_i$ to be exchanged between the cells during their
interaction (ie. species specific).
We take this as the basis for calculating forces between cells.
\begin{equation}
    (F_i, F_j) = F(x_i, x_j, v_i, v_j, \sigma_i, \sigma_j)
\end{equation}

\subsection{Domain-Cell Interactions}
\label{subsec:abstractions-domain-cell}

% \subsection{Citations}
% Citations use \verb+natbib+. The documentation may be found at
% \begin{center}
% 	\url{http://mirrors.ctan.org/macros/latex/contrib/natbib/natnotes.pdf}
% \end{center}
% 
% Here is an example usage of the two main commands (\verb+citet+ and \verb+citep+): Some people thought a thing \citep{kour2014real, hadash2018estimate} but other people thought something else \citep{kour2014fast}. Many people have speculated that if we knew exactly why \citet{kour2014fast} thought this\dots
% 
% \subsection{Figures}
% See Figure \ref{fig:fig1}. Here is how you add footnotes. \footnote{Sample of the first footnote.}
% 
% \begin{figure}
% 	\centering
% 	\fbox{\rule[-.5cm]{4cm}{4cm} \rule[-.5cm]{4cm}{0cm}}
% 	\caption{Sample figure caption.}
% 	\label{fig:fig1}
% \end{figure}
% 
% \subsection{Tables}
% See awesome Table~\ref{tab:table}.
% 
% The documentation for \verb+booktabs+ (`Publication quality tables in LaTeX') is available from:
% \begin{center}
% 	\url{https://www.ctan.org/pkg/booktabs}
% \end{center}
% 
% 
% \begin{table}
% 	\caption{Sample table title}
% 	\centering
% 	\begin{tabular}{lll}
% 		\toprule
% 		\multicolumn{2}{c}{Part}                   \\
% 		\cmidrule(r){1-2}
% 		Name     & Description     & Size ($\mu$m) \\
% 		\midrule
% 		Dendrite & Input terminal  & $\sim$100     \\
% 		Axon     & Output terminal & $\sim$10      \\
% 		Soma     & Cell body       & up to $10^6$  \\
% 		\bottomrule
% 	\end{tabular}
% 	\label{tab:table}
% \end{table}
% 
% \subsection{Lists}
% \begin{itemize}
% 	\item Lorem ipsum dolor sit amet
% 	\item consectetur adipiscing elit.
% 	\item Aliquam dignissim blandit est, in dictum tortor gravida eget. In ac rutrum magna.
% \end{itemize}


\bibliographystyle{unsrtnat}
\bibliography{references}  %%% Uncomment this line and comment out the ``thebibliography'' section below to use the external .bib file (using bibtex) .


%%% Uncomment this section and comment out the \bibliography{references} line above to use inline references.
% \begin{thebibliography}{1}

% 	\bibitem{kour2014real}
% 	George Kour and Raid Saabne.
% 	\newblock Real-time segmentation of on-line handwritten arabic script.
% 	\newblock In {\em Frontiers in Handwriting Recognition (ICFHR), 2014 14th
% 			International Conference on}, pages 417--422. IEEE, 2014.

% 	\bibitem{kour2014fast}
% 	George Kour and Raid Saabne.
% 	\newblock Fast classification of handwritten on-line arabic characters.
% 	\newblock In {\em Soft Computing and Pattern Recognition (SoCPaR), 2014 6th
% 			International Conference of}, pages 312--318. IEEE, 2014.

% 	\bibitem{hadash2018estimate}
% 	Guy Hadash, Einat Kermany, Boaz Carmeli, Ofer Lavi, George Kour, and Alon
% 	Jacovi.
% 	\newblock Estimate and replace: A novel approach to integrating deep neural
% 	networks with existing applications.
% 	\newblock {\em arXiv preprint arXiv:1804.09028}, 2018.

% \end{thebibliography}


\end{document}

